\documentclass[11pt,a4paper]{article}
%\documentclass[12pt, a4j]{jreport}
\usepackage{amsmath,amssymb,amsthm,bm,euscript}
\usepackage[whole]{bxcjkjatype}
\usepackage[]{graphicx}
\usepackage{algorithm}
\usepackage{algpseudocode}
\usepackage[toc,page]{appendix}
\usepackage{float}
\usepackage{comment}
\usepackage{indentfirst}
\usepackage{geometry}
\geometry{left=30mm,right=30mm,top=30mm,bottom=30mm}

\renewcommand{\baselinestretch}{1.5}
\newcommand{\argmax}{\mathop{\rm arg~max}\limits}
\newcommand{\argmin}{\mathop{\rm arg~min}\limits}
\newcommand{\regret}{\mathop{\rm Regret}}

\title{ベイジアンモデルを用いた時系列予測合成}
\author{\emph{慶應義塾大学経済学部四年}\\ \emph{中妻照雄研究会} \\渡邉 直斗  }
\date{}

%\theoremstyle{break}
\begin{document}

\maketitle

\begin{abstract}
近年,ベイズ的手法を用いてのモデルや予測の比較,測定や結合に関する研究が行われている.その中で多くの,予測密度のプール化の手法が考案されてきた.
\end{abstract}

\clearpage
\setcounter{tocdepth}{1}
\tableofcontents
\clearpage
\newtheorem{theo}{定理}[section]
\newtheorem{defi}[theo]{定義}
\newtheorem{lemm}[theo]{補題}
\newtheorem{Proof}{証明}

\section{イントロダクション}


\section{手法}
今回の論文では,Agent Opinion Analysisと呼ばれる統計的手法を動的に使えるように拡張した手法(McAlinn and West)を用いることによって分析を行う.
\subsection{Agent Opinion Analysis}
ここでは,基本的なAgent Opinion Analysis(West 1992)についてまとめる.この段階では,時系列解析には対応していない.\par
ベイジアン意思決定者$\mathcal{D}$は$\mathcal{J}$のエージェント$\mathcal{A}_j,(j = 1:\mathcal{J})$が提供する情報をもとに結果$y$を予測する(ここでのエージェントは、モデル、予測者などを意味する).
まず,$\mathcal{D}$は事前確率$p(y)$を持ち,エージェントは予測情報である$h_j(y)$を提供する.提供された情報セットを$\mathcal{H} = \{h_1(\cdot),...,h_j(\cdot)\}$とすると,$\mathcal{D}$は自身の事後確率$p(p|\mathcal{H})$を用いて$y$を予測するようになる.これがAgent Opinion Analysisになる.\par
しかし,各々のバイアスに加え,エージェント間の相互依存などが存在するため,簡単に$p(y,\mathcal{H})=p(y)p(\mathcal{H}|y)$を概念化できない.そこで,概念化のための関連するAgent Opinion Analysis理論(Gunest and Schervish 1985; West and Crosse 1992)についてまとめる.この理論では,$\mathcal{D}$は情報セット$\mathcal{H}$を観測する前に$p(y)$を指定すると仮定し,その期待値を$m({\bf x}) = E[\prod_{j=1:j}h_j({\b fx}_j)]$とする(ここで$\bold{x} = (x_1,...,x_j)'$はエージェントのダミー変数ベクトル).すると,以下のベイジアンモデル$p(y,\mathcal{H})$のサブセットが導かれる.
\[
p(y|\mathcal{H}) = \int_x \alpha(y|{\bf x}) \prod_{j=1:j} h_j(x_J)d\bf x
\]
ここで,$\alpha (y|\bf x)$は$\bf x$が与えられたときの$y$の条件付き確率である.これにより,潜在因子$\bf x$は次の3つの解釈ができる.

\subsection{動的連続設計}
意思決定者$\mathcal{D}$は,各時系列$t = 1,2,...$において,エージェントから予測密度を受け取り,連続的に$y_t$を予測している.例えば,$t-1$の時,意思決定者$\mathcal{D}$は,エージェントから,予測密度のデータセット$\mathcal{H} = \{h_{t1}(y_t),...,h_{tJ}(y_t)\}$を受け取り,$y_t$を予測しようとする.意思決定者が,使う情報をまとめると$\{y_{1:t-1},\mathcal{H}_{1:t}\}$と表せる.前述のモデルを動的に対応できるよう拡張すると,意思決定者$\mathcal{D}$が,$yt$を予測するための$t-1$時点での分布は以下のような形になる.
\[
p(y_t|\Phi_t,y_{1:t-1},\mathcal{H}_{1:t}) \equiv p(y_t|\Phi_t,\mathcal{H}_t) = \int \alpha_t(y_t|{\bf x}_t,\Phi_t) \prod_{j=1:J} h_tj(x_{tj})dx_{tj}
\]
ここで,${\bf x}_t = x_{t,1:J}$は,時間$t$におけるエージェント$\mathcal{J}$の状態ベクトルで,$\alpha_t(y_t|{\bf x}_t,\Phi_t)$は意思決定者$\mathcal{D}$が,$y_t$を予測するため${\bf x}_t$を与えられた時の,条件付きキャリブレーションの確率密度関数である.そして$\Phi_t$は,$\mathcal{D}$の$t-1$時点での事後分布$p(\Phi_t|y_{1:t-1},\mathcal{H}_{1:t-1})$のパラメーターをキャリブレーションする確率密度関数を定義する時変パラメーターを表す.
\[
\alpha_t(y_t|x_t,\Phi_t) = N(y_t|F'_t\theta_t,v_t) \;{\rm with}\; F_t = (1,x'_t)' \;{\rm and}\; \theta_t = (\theta_{t0},\theta_{t1},...,\theta_{tJ})'
\]

\begin{equation}
\begin{split}
y_t &= F'_t\theta_t + v_t,   v_t \sim N(0,v_t),\\
\theta_t &= \theta_{t-1} + w_t, w_t \sim N(0,v_tW_t)
\end{split}
\label{ons1}
\end{equation}















\end{document}