\documentclass[11pt,a4paper]{article}
%\documentclass[12pt, a4j]{jreport}
\usepackage{amsmath,amssymb,amsthm,bm,euscript}
\usepackage[whole]{bxcjkjatype}
\usepackage[]{graphicx}
\usepackage{algorithm}
\usepackage{algpseudocode}
\usepackage[toc,page]{appendix}
\usepackage{float}
\usepackage{comment}
\usepackage{indentfirst}
\usepackage{geometry}
\geometry{left=30mm,right=30mm,top=30mm,bottom=30mm}

\renewcommand{\baselinestretch}{1.5}
\newcommand{\argmax}{\mathop{\rm arg~max}\limits}
\newcommand{\argmin}{\mathop{\rm arg~min}\limits}
\newcommand{\regret}{\mathop{\rm Regret}}

\title{動的ベイジアン予測合成}
\author{\emph{慶應義塾大学経済学部四年}\\ \emph{中妻照雄研究会} \\渡邉 直斗  }
\date{}

%\theoremstyle{break}
\begin{document}

\maketitle

\begin{abstract}
近年,ベイズ的手法を用いてのモデルや予測の比較,測定や結合に関する研究が行われている.その中で多くの,予測密度のプール化の手法が考案されてきた.
\end{abstract}

\clearpage
\setcounter{tocdepth}{1}
\tableofcontents
\clearpage
\newtheorem{theo}{定理}[section]
\newtheorem{defi}[theo]{定義}
\newtheorem{lemm}[theo]{補題}
\newtheorem{Proof}{証明}

\section{イントロダクション}


\section{背景}
\subsection{Agent Opinion Analysis}
ここでは,基本的なAgent Opinion Analysisについてまとめる.この段階では,時系列解析には対応していない.\par
ベイジアン意思決定者$\mathcal{D}$は$\mathcal{J}$のエージェント$\mathcal{A}_j,(j = 1:\mathcal{J})$が提供する情報をもとに結果$y$を予測する(ここでのエージェントは、モデル、予測者などを意味する).
まず,$\mathcal{D}$は事前確率$p(y)$を持ち,エージェントは予測情報である$h_j(y)$を提供する.提供された情報セットを$\mathcal{H} = \{h_1(\cdot),...,h_j(\cdot)\}$とすると,$\mathcal{D}$は自身の事後確率$p(p|\mathcal{H})$を用いて$y$を予測するようになる.これがAgent Opinion Analysisになる.\par
しかし,各々のバイアスに加え,エージェント間の相互依存などが存在するため,簡単に$p(y,\mathcal{H})=p(y)p(\mathcal{H}|y)$を概念化できない.そこで,概念化のための関連するAgent Opinion Analysis理論についてまとめる.この理論では,$\mathcal{D}$は情報セット$\mathcal{H}$を観測する前に$p(y)$を指定すると仮定し,その期待値を$m({\bf x}) = E[\prod_{j=1:j}h_j({\b fx}_j)]$とする(ここで$\bold{x} = (x_1,...,x_j)'$はエージェントのダミー変数ベクトル).すると,以下のベイジアンモデル$p(y,\mathcal{H})$のサブセットが導かれる.
\[
p(y|\mathcal{H}) = \int_x \alpha(y|{\bf x}) \prod_{j=1:j} h_j(x_j)d\bf x
\]
ここで,$\alpha (y|\bf x)$は$\bf x$が与えられたときの$y$の条件付き確率である.また,潜在因子$\bf x$は次の3つの解釈ができる.



\end{document}